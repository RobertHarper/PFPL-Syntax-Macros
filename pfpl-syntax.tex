\documentclass[11pt]{article}
\usepackage[authoryear,semicolon]{natbib}
\usepackage[T1]{fontenc}
\usepackage[utf8]{inputenc}
\usepackage[english]{babel}
\usepackage{textgreek}
\usepackage[color=yellow,textwidth=1.0in]{todonotes}
\setlength{\marginparwidth}{1.25in}
\usepackage{xifthen}
\usepackage{amsmath,amssymb,amsthm,mathtools,stmaryrd}
\usepackage{mathpartir}
\usepackage{colonequals}
\usepackage{textcomp}
\usepackage[us]{optional}
\usepackage{color}
\usepackage{url}
\usepackage{hyperref}
\usepackage{doi}
\usepackage{graphics}
\usepackage{import}
\usepackage{stackengine}
\usepackage{scalerel}
\usepackage{verbatim}
\usepackage{geometry}

\usepackage[abt]{pfpl-syntax}

\allowdisplaybreaks[1]       %mildly permissible to break up displayed equations

\title{\textsf{PFPL} Syntax Package%
\footnote{\copyright{} \the\year{} Robert Harper.  All Rights Reserved.}}
\author{Robert Harper}
\date{\today}

\NewDocumentCommand{\singmap}{m m}{#1\hookrightarrow #2}
\NewDocumentCommand{\finmap}{s m m m}{\IfBooleanTF{#1}{#4_{#2}}{\singmap{#3}{#4_#3}\mid #3\in #2}}

\begin{document}

\maketitle{}

\section*{Overview\footnotemark}

Write \verb|\usepackage{pfpl-syntax}| to define syntax macros for \textsf{PFPL}.  The package provides an integrated, systematic treatment of abstract and concrete syntax for a wide range of languages.

\footnotetext{I am grateful to Kartik Singal for help with testing and improving this package.}

\section*{Abstract Binding Trees}

To typeset an abstract binding tree yourself, provide the package with the \verb|[abt]| option, and use the following commands:
\begin{itemize}
  \item \verb|\Opn{op}|: Format an operator as a keyword; use starred form for an operator whose display is to be set in math mode.  For example, \verb|\Opn{fun}| produces $\Opn{fun}$ and \verb|\Opn*{\lambda}| produces $\Opn*{\lambda}$.
  \item \verb|\Abt<parameters>[optional](arguments)|: produce abstract syntax, with all arguments typeset; the starred form omits the optional arguments.  If the parenthesized arguments are empty, then no parentheses are typeset.  For example, the command
  \begin{quote}
\begin{verbatim}
\Abt{\Opn{get}}<a>[\tau]
\end{verbatim}
  \end{quote}
  produces $\Abt{\Opn{get}}<a>[\tau]$, and the command
  \begin{quote}
\begin{verbatim}
\Abt{\Opn{set}}[\tau]<a>(e)
\end{verbatim}
  \end{quote}
  produces $\Abt{\Opn{set}}<a>[\tau](e)$.  The starred forms produce $\Abt*{\Opn{get}}<a>[\tau]$, and $\Abt*{\Opn{set}}<a>[\tau](e)$, respectively.

  \item \verb|\Abs<symbols>(variables){abt}|: produce an abstractor over symbols and/or variables within an ABT.  For example, the command
  \begin{quote}
\begin{verbatim}
\Abt{\Opn*{\lambda}}[\tau](\Abs(x){e})
\end{verbatim}
  \end{quote}
  produces $\Abt{\Opn*{\lambda}}[\tau](\Abs(x){e})$ and the starred form produces $\Abt*{\Opn*{\lambda}}[\tau](\Abs(x){e})$.  Similarly, the command
  \begin{quote}
\begin{verbatim}
\Abt{\Opn{dcl}}[\tau;\rho](e;\Abs<a>{m})
\end{verbatim}
  \end{quote}
  produces $\Abt{\Opn{dcl}}[\tau;\rho](e;\Abs<a>{m})$ and the starred form produces $\Abt*{\Opn{dcl}}[\tau\,\rho](e;\Abs<a>{m})$.
\end{itemize}
The package option \verb|[sf]| indicates that operations are to be set in sans serif font, and the package option \verb|[sc]| indicates that they are to be set in small capitals.  These forms are not ordinarily used directly, they are rather for authors to define their own syntax macros, as illustrated for \textsf{PFPL} in the next section.  The package option \verb|[oldstyle]| uses the notational conventions from the second edition in which symbol parameters to operators are typeset in square brackets, and their minor arguments are typeset in curly braces.

The command \verb|\Sub{e}{x}{e'}| expands to $\Sub{e}{x}{e'}$, and \verb|\Sub*{e}{x}{e'}| expands to $\Sub*{e}{x}{e'}$, providing pre- and post-fix notation for substitution.  It should only be used when it is defined on $\alpha$-equivalence of abt's and abs's so that it is always well-defined and avoids capture.

\section*{Language-Specific Definitions}

The commands for formatting the many languages considered in \textsf{PFPL} are formulated in a series of figures organized around type structure.  The tables display the abstract and concrete syntax and the literal command used to create them (with starred forms for the concrete syntax).

When $I$ is a finite set of indices, the notation $\tau_I$ stands for the finite map $\finmap{I}{i}{\tau}$ and $e_I$ stands for $\finmap{I}{i}{e}$.

\clearpage
\newgeometry{hmargin=14mm}

\begin{figure}
  \begin{small}
    \begin{displaymath}
      \begin{array}{l@{\qquad}l@{\qquad}l@{\qquad}l}
        \arrTy{\tau_1}{\tau_2} & \arrTy*{\tau_1}{\tau_2} & \verb|\arrTy{\tau_1}{\tau_2}| \\
        \lamEx{x}{e}           & \lamEx*{x}{e}           & \verb|\lamEx{x}{e}|           \\
        \appEx{e_1}{e_2}       & \appEx*{e_1}{e_2}       & \verb|\appEx{e_1}{e_2}|
      \end{array}
    \end{displaymath}
  \end{small}

  \caption{Function Types}
  \label{fig:function}
\end{figure}

\begin{figure}

  \begin{small}
    \begin{displaymath}
      \begin{array}{l@{\qquad}l@{\qquad}l@{\qquad}l}
        \unitTy                           & \unitTy*                 & \verb|\unitTy|                                \\
        \unitEx                           & \unitEx*                 & \verb|\unitEx|                                \\[1ex]

        \prodTy{\tau_1}{\tau_2}           & \prodTy*{\tau_1}{\tau_2} & \verb|\prodTy{\tau_1}{\tau_2}|                \\
        \projEx<i>{e}                     & \projEx*<i>{e}           & \verb|\projEx<i>{e}| \quad (i=1,2)            \\
        \pairEx[\tau_1][\tau_2]{e_1}{e_2} & \pairEx*{e_1}{e_2}       & \verb|\pairEx[\tau_1][\tau_2]{e_1}{e_2}|      \\[1ex]

        \vprodTy<I><i>{\tau}              & \vprodTy*{\tau}          & \verb|\vprodTy<I><i>{\tau}|                   \\
        \vtupleEx<I>[\tau]{e}             & \vtupleEx*<I>{e}         & \verb|\vtupleEx<I>[\tau]{e}|                  \\
        \vprojEx<I><i>[\tau]{e}           & \vprojEx*<i>{e}          & \verb|\vprojEx<I><i>[\tau]{e}| \quad (i\in I)
      \end{array}
    \end{displaymath}
  \end{small}

  \caption{Product Types}
  \label{fig:product}
\end{figure}

\begin{figure}

  \begin{small}
    \begin{displaymath}
      \begin{array}{l@{\qquad}l@{\qquad}l@{\qquad}l}
        \voidTy                           & \voidTy*                & \verb|\voidTy|                                    \\
        \absurdEx[\rho]{e}                & \absurdEx*{e}           & \verb|\absurdEx[\rho]{e}|                         \\[1ex]

        \sumTy{\tau_1}{\tau_2}            & \sumTy*{\tau_1}{\tau_2} & \verb|\sumTy{\tau_1}{\tau_2}|                     \\
        \injEx<i>[\tau_1][\tau_2]{e}      & \injEx*<i>{e}           & \verb|\injEx<i>[\tau_1][\tau_2]{e}| \quad (i=1,2) \\
        \caseEx[\tau][\rho]{e}{x}{e'}     & \caseEx*{e}{x}{e'}      & \verb|\caseEx[\tau][\rho]{e}{x}{e'}|              \\[1ex]

        \boolTy                           & \boolTy*                & \verb|\boolTy|                                    \\
        \trueEx                           & \trueEx*                & \verb|\trueEx|                                    \\
        \falseEx                          & \falseEx*               & \verb|\falseEx|                                   \\
        \ifEx[\rho]{e}{e_1}{e_2}          & \ifEx*{e}{e_1}{e_2}     & \verb|\ifEx[\rho]{e}{e_1}{e_2}|                   \\[1ex]

        \vsumTy<I><i>{\tau}               & \vsumTy*<I><i>{\tau}    & \verb|\vsumTy<I><i>{\tau}|                        \\
        \vinjEx<I><i>[\tau]{e}            & \vinjEx*<I><i>{e}       & \verb|\vinjEx<I><i>[\tau]{e}| \quad (i\in I)      \\
        \vcaseEx<I>[\tau][\rho]{e}{x}{e'} & \vcaseEx*<I>{e}{x}{e'}  & \verb|\vcaseEx<I>[\tau][\rho]{e}{x}{e'}|
      \end{array}
    \end{displaymath}
  \end{small}

  \caption{Sum Types}
  \label{fig:sums}
\end{figure}


\begin{figure}

  \begin{small}
    \begin{displaymath}
      \begin{array}{l@{\qquad}l@{\qquad}l@{\qquad}l}
        \indTy{t}{\tau}                      & \indTy*{t}{\tau}                    & \verb|\indTy{t}{\tau}|                         \\
        \inEx[t][\tau]{e}                    & \inEx*{e}                           & \verb|\inEx[t][\tau]{e}|                       \\
        \recEx[t][\tau][\rho]{e}{x}{e'}      & \recEx*{e}{x}{e'}                   & \verb|\recEx[t][\tau][\rho]{e}{x}{e'}|         \\[1ex]

        \natTy                               & \natTy*                             & \verb|\natTy|                                  \\
        \zeroEx                              & \zeroEx*                            & \verb|\zeroEx|                                 \\
        \succEx{e}                           & \succEx*{e}                         & \verb|\succEx{e}|                              \\
        \natrecEx[\rho]{e}{e_0}{x}{y}{e_1}   & \natrecEx*[\rho]{e}{e_0}{x}{y}{e_1} & \verb|\natrecEx[\rho]{e}{e_0}{x}{y}{e_1}|      \\
        \natitEx[\rho]{e}{e_0}{x}{e_1}       & \natitEx*{e}{e_0}{x}{e_1}           & \verb|\natitEx[\rho]{e}{e_0}{x}{e_1}|          \\
        \ifzEx[\rho]{e}{e_0}{x}{e_1}         & \ifzEx*{e}{e_0}{x}{e_1}             & \verb|\ifzEx[\rho]{e}{e_0}{x}{e_1}|            \\[1ex]

        \listTy{\tau}                        & \listTy*{\tau}                      & \verb|\listTy{\tau}|                           \\
        \nilEx[\tau]                         & \nilEx*                             & \verb|\nilEx[\tau]|                            \\
        \consEx[\tau]{e_1}{e_2}              & \consEx*{e_1}{e_2}                  & \verb|\consEx[\tau]{e_1}{e_2}|                 \\
        \listrecEx[\rho]{e}{e_1}{x}{y}{e_2}  & \listrecEx*{e}{e_1}{x}{y}{e_2}      & \verb|\listrecEx[\rho]{e}{e_1}{x}{y}{e_2}|     \\
        \listcaseEx[\rho]{e}{e_1}{x}{y}{e_2} & \listcaseEx*{e}{e_1}{x}{y}{e_2}     & \verb|\listcaseEx[\rho]{e}{e_1}{x}{y}{e_2}|    \\[1ex]

        \coiTy{t}{\tau}                      & \coiTy*{t}{\tau}                    & \verb|\coiTy{t}{\tau}|                         \\
        \outEx[t][\tau]{e}                   & \outEx*{e}                          & \verb|\outEx[t][\tau]{e}|                      \\
        \genEx[t][\tau][\rho]{e}{x}{e'}    & \genEx*{e}{x}{e'}                   & \verb|\genEx[t][\tau][\rho]{e}{x}{e'}|       \\[1ex]

        \conatTy                             & \conatTy*                           & \verb|\conatTy|                                \\
        \streamTy{\tau}                      & \streamTy*{\tau}                    & \verb|\streamTy{\tau}|
      \end{array}
    \end{displaymath}
  \end{small}

  \caption{Inductive and Coinductive Types}
  \label{fig:icoi}
\end{figure}

\begin{figure}

  \begin{small}
    \begin{displaymath}
      \begin{array}{l@{\qquad}l@{\qquad}l@{\qquad}l}
        \allTy{t}{\tau}                     & \allTy*{t}{\tau}                     & \verb|\allTy{t}{\tau}|                     \\
        \tlamEx{t}[\tau]{e}                 & \tlamEx*{t}{e}                       & \verb|\tlamEx{t}[\tau]{e}|                 \\
        \tapEx[t][\tau]{e}{\rho}          & \tapEx*{e}{\rho}                   & \verb|\tapEx[t][\tau]{e}{\rho}|          \\[1ex]
        \someTy{t}{\tau}                    & \someTy*{t}{\tau}                    & \verb|\someTy{t}{\tau}|                    \\
        \packEx[t][\tau]{\rho}{e}           & \packEx*{\rho}{e}                    & \verb|\packEx[t][\tau]{\rho}{e}|           \\
        \openEx[t][\tau][\rho]{e}{t}{x}{e'} & \openEx*[t][\tau][\rho]{e}{t}{x}{e'} & \verb|\openEx[t][\tau][\rho]{e}{t}{x}{e'}|
      \end{array}
    \end{displaymath}
  \end{small}

  \caption{Polymorphic Types}
  \label{fig:poly}
\end{figure}

\begin{figure}

  \begin{small}
    \begin{displaymath}
      \begin{array}{l@{\qquad}l@{\qquad}l@{\qquad}l}
        \ansTy                         & \ansTy*             & \verb|\ansTy|                         \\
        \yesEx                         & \yesEx*             & \verb|\yesEx|                         \\
        \noEx                          & \noEx*              & \verb|\noEx|                          \\[1ex]

        \contTy{\tau}                  & \contTy*{\tau}      & \verb|\contTy{\tau}|                  \\
        \contEx{k}                     & \contEx*{k}         & \verb|\contEx{k} |                    \\
        \letccEx[\tau]{x}{e}           & \letccEx*{x}{e}     & \verb|\letccEx[\tau]{x}{e}|           \\
        \throwEx[\tau][\rho]{e_1}{e_2} & \throwEx*{e_1}{e_2} & \verb|\throwEx[\tau][\rho]{e_1}{e_2}| \\[1ex]

        \empStk[\tau]                  & \empStk*[\tau]      & \verb|\empStk[\tau]|                  \\
        \extStk[\tau_1][\tau_2]{k}{f}  & \extStk*{k}{f}      & \verb|\extStk[\tau_1][\tau_2]{k}{f}|
      \end{array}
    \end{displaymath}
  \end{small}

  \caption{Continuation Types}
  \label{fig:cont}
\end{figure}

\begin{figure}

  \begin{small}
    \begin{displaymath}
      \begin{array}{l@{\qquad}l@{\qquad}l@{\qquad}l}
        \parrTy{\tau_1}{\tau_2}         & \parrTy*{\tau_1}{\tau_2} & \verb|\parrTy{\tau_1}{\tau_2}|          \\
        \funEx[\tau_1][\tau_2]{f}{x}{e} & \funEx*{f}{x}{e}         & \verb|\funEx[\tau_1][\tau_2]{f}{x}{e}|  \\
        \papEx{e_1}{e_2}                & \papEx*{e_1}{e_2}        & \verb|\papEx[\tau_1][\tau_2]{e_1}{e_2}| \\[1ex]

        \fixEx[\tau]{x}{e}              & \fixEx*[\tau]{x}{e}      & \verb|\fixEx[\tau]{x}{e}|               \\[1ex]

        \recTy{t}{\tau}                 & \recTy*{t}{\tau}         & \verb|\recTy{t}{\tau}|                  \\
        \foldEx[t][\tau]{e}             & \foldEx*{e}              & \verb|\foldEx[t][\tau]{e}|              \\
        \unfoldEx[t][\tau]{e}           & \unfoldEx*{e}            & \verb|\unfoldEx[t][\tau]{e}|            \\[1ex]

        \selfTy{\tau}                   & \selfTy*{\tau}           & \verb|\selfTy{\tau}|                    \\
        \unrollEx[\tau]{e}              & \unrollEx*{e}            & \verb|\unrollEx[\tau]{e}|               \\
        \selfEx[\tau]{x}{e}             & \selfEx*{x}{e}           & \verb|\selfEx[\tau]{x}{e}|              \\[1ex]

        \ulamEx{x}{M}                   & \ulamEx*{x}{M}           & \verb|\ulamEx{x}{M}|                    \\
        \uapEx{M_1}{M_2}                & \uapEx*{M_1}{M_2}        & \verb|\uapEx{M_1}{M_2}|                 \\
        \uIEx                           & \uIEx*                   & \verb|\uIEx|                            \\
        \uKEx                           & \uKEx*                   & \verb|\uKEx|                            \\
        \uSEx                           & \uSEx*                   & \verb|\uSEx|                            \\
        \uBEx                           & \uBEx*                   & \verb|\uBEx|
      \end{array}
    \end{displaymath}
  \end{small}

  \caption{Partial and Recursive Types}
  \label{fig:recursive}
\end{figure}

\begin{figure}

  \begin{small}
    \begin{displaymath}
      \begin{array}{l@{\qquad}l@{\qquad}l@{\qquad}l}
        \suspTy{\tau}           & \suspTy*{\tau}           & \verb|\suspTy{\tau}|           \\
        \suspEx[\tau]{e}        & \suspEx*[\tau]{e}        & \verb|\suspEx[\tau]{e}|        \\
        \forcesuspEx[\tau]{e}   & \forcesuspEx*[\tau]{e}   & \verb|\forcesuspEx[\tau]{e}|   \\[1ex]

        \compTy{\tau}           & \compTy*{\tau}           & \verb|\compTy{\tau}|           \\
        \compEx{m}              & \compEx*{m}              & \verb|\compEx{m}|              \\
        \retEx[\tau]{e}         & \retEx*[\tau]{e}         & \verb|\retEx[\tau]{e}|         \\
        \bndEx[\tau]{e_1}{x}{e_2} & \bndEx*[\tau]{e_1}{x}{e_2} & \verb|\bndEx[\tau]{e_1}{x}{e_2}|
    \end{array}
    \end{displaymath}
  \end{small}

  \caption{Suspension and Computation Types}
  \label{fig:comp-susp}
\end{figure}

\begin{figure}

  \begin{small}
    \begin{displaymath}
      \begin{array}{l@{\qquad}l@{\qquad}l@{\qquad}l}
        \cmdTy{\tau}               & \cmdTy*{\tau}               & \verb|\cmdTy{\tau}|               \\
        \cmdEx[\tau]{m}            & \cmdEx*{m}                  & \verb|\cmdEx[\tau]{m}|            \\[1ex]
        \retCmd[\tau]{e}           & \retCmd*{e}                 & \verb|\retCmd[\tau]{e}|           \\
        \bndCmd{e}{x}{m}           & \bndCmd*{e}{x}{m}           & \verb|\bndCmd{e}{x}{m}|           \\
        \dclCmd{e}{a}{m}           & \dclCmd*{e}{a}{m}           & \verb|\dclCmd{e}{a}{m}|           \\[1ex]
        \refTy{\tau}               & \refTy*{\tau}               & \verb|\refTy{\tau}| \\
        \refEx{a}                  & \refEx*{a}                  & \verb|\refEx{a}|                  \\[1ex]
        \getCmd{a}                 & \getCmd*{a}                 & \verb|\getCmd{a}|                 \\
        \getrefCmd[\tau]{e}        & \getrefCmd*{e}              & \verb|\getrefCmd[\tau]{e}|        \\
        \setCmd{a}{e}              & \setCmd*{a}{e}              & \verb|\setCmd{a}{e}|              \\
        \setrefCmd[\tau]{e_1}{e_2} & \setrefCmd*[\tau]{e_1}{e_2} & \verb|\setrefCmd[\tau]{e_1}{e_2}|
      \end{array}
    \end{displaymath}
  \end{small}

  \caption{Command Types}
  \label{fig:cmd}
\end{figure}


\begin{figure}

  \begin{small}
    \begin{displaymath}
      \begin{array}{l@{\qquad}l@{\qquad}l@{\qquad}l}
        \staticLock                    & \staticLock*         & \verb|\staticLock|                              \\
        \univSg                        & \univSg*             & \verb|\univSg|                                  \\
        \valSg{\tau}                   & \valSg*{\tau}        & \verb|\valSg{\tau}|                             \\[1ex]
        \extSg{S}{M}                   & \extSg*{S}{M}        & \verb|\extSg{S}{M}|                             \\
        \inMd[S][M]{M'}                & \inMd*{M'}           & \verb|\inMd[S][M]{M'}|                          \\
        \outMd[S][M]{M'}               & \outMd*{M'}          & \verb|\outMd[S][M]{M'}|                         \\[1ex]
        \compSg{S}                     & \compSg*{S}          & \verb|\compSg{S}|                               \\[1ex]
        \piSg{S_1}{X}{S_2}             & \piSg*{S_1}{X}{S_2}  & \verb|\piSg{S_1}{X}{S_2}|                       \\
        \funMd{S_1}{X}{M_2}            & \funMd*{S_1}{X}{M_2} & \verb|\funMd{S_1}{X}{M_2}|                      \\
        \instMd[S_1][X][S_2]{M_1}{M_2} & \instMd*{M_1}{M_2}   & \verb|\instMd[S_1][X][S_2]{M_1}{M_2}|           \\[1ex]
        \sigSg{S_1}{X}{S_2}            & \sigSg*{S_1}{X}{S_2} & \verb|\sigSg{S_1}{X}{S_2}|                      \\
        \strMd[S_1][X][S_2]{M_1}{M_2}  & \strMd*{M_1}{M_2}    & \verb|\strMd[S_1][X][S_2]{M_1}{M_2}|            \\
        \projMd<i>[S_1][X][S_2]{M}     & \projMd*<i>{M}       & \verb|\projMd<i>[S_1][X][S_2]{M}| \quad (i=1,2)
      \end{array}
    \end{displaymath}
  \end{small}

  \caption{Module Types}
  \label{fig:mod}
\end{figure}

\begin{figure}

  \begin{small}
    \begin{displaymath}
      \begin{array}{l@{\qquad}l@{\qquad}l@{\qquad}l}
        \unitPr            & \unitPr*            & \verb|\unitPr |           \\
        \parPr{p_1}{p_2}     & \parPr*{p_1}{p_2}     & \verb|\parPr{p_1}{p_2}|     \\[1ex]
        \nullPr            & \nullPr*            & \verb|\nullPr|            \\
        \choosePr{p_1}{p_2}  & \choosePr*{p_1}{p_2}  & \verb|\choosePr{p_1}{p_2}|  \\[1ex]
        \quePr{a}{p}         & \quePr*{a}{p}         & \verb|\quePr{a}{p}|         \\
        \sigPr{a}{p}         & \sigPr*{a}{p}         & \verb||                     \\[1ex]
        \rcvPr{a}{x}{p}      & \rcvPr*{a}{x}{p}      & \verb|\sigPr{a}{p}|         \\
        \sndPr{a}{e}{p}      & \sndPr*{a}{e}{p}      & \verb|\sndPr{a}{e}{p}|      \\[1ex]
        \asndPr{a}{e}        & \asndPr*{a}{e}        & \verb|\asndPr{a}{e}|        \\
        \syncPr{\varepsilon} & \syncPr*{\varepsilon} & \verb|\syncPr{\varepsilon}| \\[1ex]
        \newPr{a}{p}         & \newPr*{a}{p}         & \verb|\newPr{a}{p}|         \\[1ex]
        \silAc               & \silAc*               & \verb|\silAc|               \\
        \sigAc{a}            & \sigAc*{a}            & \verb|\sigAc{a}|            \\
        \queAc{a}            & \queAc*{a}            & \verb|\queAc{a}|            \\
        \sndAc{a}{e}         & \sndAc*{a}{e}         & \verb|\sndAc{a}{e}|         \\
        \rcvAc{a}{e}         & \rcvAc*{a}{e}         & \verb|\rcvAc{a}{e}|         \\
        \anyAc{e}            & \anyAc*{e}            & \verb|\anyAc{e}|         \\
        \emitAc{e}           & \emitAc*{e}           & \verb|\emitAc{e}|
      \end{array}
    \end{displaymath}
  \end{small}

  \caption{Proceses}
  \label{fig:proc}
\end{figure}

\begin{figure}

  \begin{small}
    \begin{displaymath}
      \begin{array}{l@{\qquad}l@{\qquad}l@{\qquad}l}
        \topTy                                        & \topTy*                   & \verb|\topTy| \\
        \topEx                                        & \topEx*                   & \verb|\topEx| \\
        \checkEx[\negTy{\rho}]{e}{e'}                 & \checkEx*{e}{e'}          & \verb|\checkEx[\rho]{e}{e'}| \\
        \freeTy{\posTy{\tau}}                         & \freeTy*{\posTy{\tau}}    & \verb|\freeTy{\tau}|                         \\
        \freeEx[\posTy{\tau}]{v}                      & \freeEx*[\posTy{\tau}]{v} & \verb|\freeEx[\tau]{v}|                      \\
        \fletEx[\negTy{\tau}][\negTy{\rho}]{e}{x}{e'} & \fletEx*{e}{x}{e'}        & \verb|\fletEx[\tau][\rho]{e}{x}{e'}| \\
        \thunkTy{\negTy{\tau}}                        & \thunkTy*{\negTy{\tau}}   & \verb|\thunkTy{\tau}|                        \\
        \thunkEx[\negTy{\tau}]{e}                     & \thunkEx*{e}              & \verb|\thunkEx[\tau]{e}|                     \\
        \forceEx[\negTy{\tau}]{v}                     & \forceEx*{v}              & \verb|\forceEx[\tau]{v}|
      \end{array}
    \end{displaymath}
  \end{small}

  \caption{Polarized Types}
  \label{fig:pol}
\end{figure}

\begin{figure}

  \begin{small}
    \begin{displaymath}
      \begin{array}{l@{\qquad}l@{\qquad}l@{\qquad}l}
        \tensorTy{\tau_1}{\tau_2}                      & \tensorTy*{\tau_1}{\tau_2} & \verb|\tensorTy{\tau_1}{\tau_2}|                      \\
        \tensorEx{v_1}{v_2}                            & \tensorEx*{v_1}{v_2}       & \verb|\tensorEx{v_1}{v_2}|                            \\
        \splitEx[\tau_1][\tau_2][\rho]{v}{x_1}{x_2}{e} & \splitEx*{v}{x_1}{x_2}{e}  & \verb|\splitEx[\tau_1][\tau_2][\rho]{v}{x_1}{x_2}{e}| \\
        \bothTy{\tau_1}{\tau_2}                        & \bothTy*{\tau_1}{\tau_2}   & \verb|\bothTy{\tau_1}{\tau_2}|                        \\
        \bothEx{e_1}{e_2}                              & \bothEx*{e_1}{e_2}         & \verb|\bothEx{e_1}{e_2}|                              \\
        \parEx{e}{x}{y}{e'}                            & \parEx*{e}{x}{y}{e'}       & \verb|\parEx{e}{x}{y}{e'}|
      \end{array}
    \end{displaymath}
  \end{small}

  \caption{Parallel Types}
  \label{fig:par}
\end{figure}

\begin{figure}

  \begin{small}
    \begin{displaymath}
      \begin{array}{l@{\qquad}l@{\qquad}l@{\qquad}l}
        \trueProp                                       & \trueProp*                       & \verb|\trueProp|                                       \\
        \trueIPf                                        & \trueIPf*                        & \verb|\trueIPf|                                        \\[1ex]

        \andProp{\phi_1}{\phi_2}                        & \andProp*{\phi_1}{\phi_2}        & \verb|\andProp{\phi_1}{\phi_2}|                        \\
        \andIPf{\pi_1}{\pi_2}                           & \andIPf*{\pi_1}{\pi_2}           & \verb|\andIPf{\pi_1}{\pi_2}|                           \\
        \andEPf[\phi_1][\phi_2]<i>{\pi}                 & \andEPf*<i>{\pi}                 & \verb|\andEPf[\phi_1][\phi_2]<i>{\pi}|                 \\[1ex]

        \falseProp                                      & \falseProp*                      & \verb|\falseProp|                                      \\
        \falseIPf[\phi]{\pi}                            & \falseIPf*{\pi}                  & \verb|\falseIPf[\phi]{\pi}|                            \\
        \falseEPf[\phi]{\pi}                            & \falseEPf*{\pi}                  & \verb|\falseEPf[\phi]{\pi}|                            \\[1ex]

        \orProp{\phi_1}{\phi_2}                         & \orProp*{\phi_1}{\phi_2}         & \verb|\orProp{\phi_1}{\phi_2}|                         \\
        \orIPf[\phi_1][\phi_2]<i>{\pi}                  & \orIPf*<i>{\pi}                  & \verb|\orIPf[\phi_1][\phi_2]<i>{\pi}|                  \\
        \orEPf[\phi_1][\phi_2]{\pi}{x}{\pi_1}{x}{\pi_2} & \orEPf*{\pi}{x}{\pi_1}{x}{\pi_2} & \verb|\orEPf[\phi_1][\phi_2]{\pi}{x}{\pi_1}{x}{\pi_2}| \\[1ex]

        \impProp{\phi_1}{\phi_2}                        & \impProp*{\phi_1}{\phi_2}        & \verb|\impProp{\phi_1}{\phi_2}|                        \\
        \impIPf[\phi_1][\phi_2]{x}{\pi_2}               & \impIPf*{x}{\pi_2}               & \verb|\impIPf[\phi_1][\phi_2]{x}{\pi_2}|               \\
        \impEPf[\phi_1][\phi_2]{\pi}{\pi_1}             & \impEPf*{\pi}{\pi_2}             & \verb|\impEPf[\phi_1][\phi_2]{\pi}{\pi_1}|             \\[1ex]

        \notProp{\phi}                                  & \notProp*{\phi}                  & \verb|\notProp{\phi}|                                  \\
        \notIPf[\phi]{x}{\pi}                           & \notIPf*{x}{\pi}                 & \verb|\notIPf[\phi]{x}{\pi}|                           \\
        \notEPf[\phi]{\pi}{\pi_2}                       & \notEPf*{\pi}{\pi_2}             & \verb|\notEPf[\phi]{\pi}{\pi_2}|
      \end{array}
    \end{displaymath}
  \end{small}

  \caption{Propositions and Proofs}
  \label{fig:prop}
\end{figure}

\restoregeometry

\end{document}
