\documentclass[11pt,twoside]{article}
\usepackage[authoryear,semicolon]{natbib}
\usepackage[T1]{fontenc}
\usepackage[utf8]{inputenc}
\usepackage[english]{babel}
\usepackage{textgreek}
\usepackage{fullpage}
\usepackage[color=yellow,textwidth=1.0in]{todonotes}
\setlength{\marginparwidth}{1.25in}
\usepackage{xifthen}
\usepackage{amsmath,amssymb,amsthm,mathtools,stmaryrd}
\usepackage{proof,mathpartir}
\usepackage{colonequals}
\usepackage{verbatim}
\usepackage{comment}
\usepackage{textcomp}
\usepackage[us]{optional}
\usepackage{color}
\usepackage{url}
\usepackage{hyperref}
\usepackage{doi}
\usepackage{graphics}
\usepackage{import}
\usepackage{stackengine}
\usepackage{scalerel}

\usepackage{pfpl-syntax}

\allowdisplaybreaks[1]       %mildly permissible to break up displayed equations

% \subimport{../}{generic-defns}
% \subimport{../}{supp-defns}
% \setboolean{live}{true}

\ProvideDocumentCommand{\PFPL}{}{\textsf{PFPL}}
\ProvideDocumentCommand{\eqdef}{}{\mathrel{\triangleq}}
\ProvideDocumentCommand{\finset}{m}{\underline{#1}}

\title{\PFPL{} Syntax Master Chart%
\footnote{\copyright{} \the\year{} Robert Harper.  All Rights Reserved.}}
\author{Robert Harper}
\date{\today}

\begin{document}

\maketitle{}

Yet another test.  Test test.

% Finite sets: $\finset{n}\eqdef{}\{\,\kw{1},\kw{2},\dots,\kw{n}\,\}$.

Answer type:
\begin{displaymath}
  \begin{array}{l@{\quad}l@{\quad}l}
    \ansty & \ansty* \\
    \yesex & \yesex* \\
    \noex & \noex*
  \end{array}
\end{displaymath}

Basic types:
\begin{displaymath}
  \begin{array}{l@{\quad}l@{\quad}l}
    \boolty & \boolty* \\
    \trueex & \trueex* \\
    \falseex & \falseex* \\
    \ifex{e}{e_1}{e_2} & \ifex*{e}{e_1}{e_2} \\[1ex]

    \arrty{\tau_1}{\tau_2} & \arrty*{\tau_1}{\tau_2} \\
    \lamex{x}{e} & \lamex*{x}{e} \\
    \appex{e_1}{e_2} & \appex*{e_1}{e_2} \\[1ex]

    \unitty{} & \unitty*{} \\
    \unitex{} & \unitex*{} \\[1ex]

    \prodty{\tau_1}{\tau_2} & \prodty*{\tau_1}{\tau_2} \\
    \projex<i>{e} & \projex*<i>{e} & (i=\kw{1},\kw{2}) \\
    \pairex{e_1}{e_2} & \pairex*{e_1}{e_2} \\[1ex]

    \voidty & \voidty* \\
    \absurdex[\rho]{e} & \absurdex*{e} \\[1ex]

    \sumty{\tau_1}{\tau_2} & \sumty*{\tau_1}{\tau_2} \\
    \injex<i>[\tau_1][\tau_2]{e} & \injex*<i>{e} & (i=\kw{1},\kw{2}) \\
    \caseex[\tau][\rho]{e}{x}{e'} & \multicolumn{2}{l}{\!\!\!\caseex*{e}{x}{e'}}   % implicitly indexed branches

  \end{array}
\end{displaymath}

Variadic forms:\footnote{All operators are indexed by finite sets $I$ of constants of sort \kw{lbl}.  Notation: $\tau_I\eqdef \finmap{I}{i}{\tau}$.}
\begin{displaymath}
  \begin{array}{l@{\quad}l@{\quad}l}
    \vprodty<I><i>{\tau} & \vprodty*{\tau} \\
    \vtupleex<I>[\tau]{e} & \vtupleex*<I>{e} \\
    \vprojex<I><i>[\tau]{e} & \vprojex*<i>{e} & (i\in I) \\[1ex]

    \vsumty<I><i>{\tau}  & \vsumty*<I><i>{\tau} \\
    \vinjex<I><i>[\tau]{e} & \vinjex*<I><i>{e} & (i\in I) \\
    \vcaseex<I>[\rho][\tau]{e}{x}{e'}  & \vcaseex*<I>{e}{x}{e'}
  \end{array}
\end{displaymath}

Continuations:
\begin{displaymath}
  \begin{array}{l@{\quad}l@{\quad}l}
    \contty{\tau}     & \contty*{\tau} \\
    \contex{k}        & \contex*{k} \\
    \letccex[\tau]{x}{e} & \letccex*{x}{e} \\
    \throwex[\tau][\rho]{e_1}{e_2} & \throwex*{e_1}{e_2} \\
    \empstk[\tau]     & \empstk*[\tau] \\
    \extstk{k}{f}     & \extstk*{k}{f} 
  \end{array}
\end{displaymath}

Polymorphism:
\begin{displaymath}
  \begin{array}{l@{\quad}l@{\quad}l}
    \Allty{t}{\tau} & \Allty*{t}{\tau} \\
    \Lamex{t}[\tau]{e} & \Lamex*{t}{e} \\
    \Appex[t][\tau]{e}{\sigma} & \Appex*{e}{\sigma}\\[1ex]
    \Somety{t}{\tau}  & \Somety*{t}{\tau} \\
    \Packex[t][\tau]{\rho}{e} & \Packex*{\rho}{e} \\
    \Openex[t][\tau]{e}[\rho]{x}{e'} & \Openex*[t][\tau]{e}[\rho]{x}{e'}
  \end{array}
\end{displaymath}

Inductive and co-inductive types:\footnote{Use of $\Opn{fold}$ and $\Opn{unfold}$ may conflict with recursive types when both are present.}
\begin{displaymath}
  \begin{array}{l@{\quad}l@{\quad}l}
    \indty{t}{\tau} & \indty*{t}{\tau} \\
    \foldex[t][\tau]{e} & \foldex*{e} \\
    \recex[t][\tau][\rho]{e}{x}{e'} & \recex*{e}{x}{e'} \\[1ex]
    \coity{t}{\tau} & \coity*{t}{\tau} \\
    \unfoldex[t][\tau]{e} & \unfoldex*{e} \\
    \genex[t][\tau][\sigma]{e}{x}{e'} & \genex*{e}{x}{e'}
  \end{array}
\end{displaymath}

Recursive types:
\begin{displaymath}
  \begin{array}{l@{\quad}l@{\quad}l}

    \parrty{\tau_1}{\tau_2} & \parrty*{\tau_1}{\tau_2} \\
    \funex{f}{x}{e} & \funex*{f}{x}{e} \\
    \appex{e_1}{e_2} & \appex*{e_1}{e_2} \\[1ex]

    \fixex[\tau]{x}{e} & \fixex*[\tau]{x}{e}  \\[1ex]

    \recty{t}{\tau} & \recty*{t}{\tau} \\
    \foldex[t][\tau]{e}  & \foldex*{e} \\
    \unfoldex[t][\tau]{e} & \unfoldex*{e} \\[1ex]

    \selfty{\tau}  & \selfty*{\tau} \\
    \rollex[\tau]{e} & \rollex*{e} \\
    \selfex[\tau]{x}{e} & \selfex*{x}{e}
  \end{array}
\end{displaymath}

Untyped $\lambda$-calculus:
\begin{displaymath}
  \begin{array}{l@{\quad}l@{\quad}l}
    \ulamex{x}{M} & \ulamex*{x}{M} \\
    \uapex{M_1}{M_2} & \uapex*{M_1}{M_2} \\
    \uIex & \uIex* \\
    \uKex & \uKex* \\
    \uSex & \uSex& \\
    \uBex & \uBex*
  \end{array}
\end{displaymath}

Commands:\footnote{Symbols $a$ are constants of sort $\kw{loc}$.}
\begin{displaymath}
  \begin{array}{l@{\quad}l@{\quad}l}
    \cmdty{\tau}  & \cmdty*{\tau} \\
    \cmdex[\tau]{m} & \cmdex*{m} \\[1ex]
    \retcmd[\tau]{e} & \retcmd*{e} \\
    \bndcmd{e}{x}{m} & \bndcmd*{e}{x}{m} \\
    \dclcmd{e}{a}{m} & \dclcmd*{e}{a}{m} \\[1ex]
    \refex{a} & \refex*{a} \\[1ex]
    \getcmd{a} & \getcmd*{a} \\
    \getrefcmd[\tau]{e} & \getrefcmd*{e} \\
    \setcmd{a}{e} & \setcmd*{a}{e} \\
    \setrefcmd[\tau]{e_1}{e_2} & \setrefcmd*[\tau]{e_1}{e_2}
 \end{array}
\end{displaymath}

Modules:
\begin{displaymath}
  \begin{array}{l@{\quad}l@{\quad}l}
    \univsg  & \univsg* \\
    \valmod{\tau} & \valmod*{\tau} \\
    \extsg{S}{M} & \extsg*{S}{M} \\
    \compsg{S}  & \compsg*{S} \\
    \pisg{S_1}{X}{S_2} & \pisg*{S_1}{X}{S_2} \\
    \sigsg{S_1}{X}{S_2} & \sigsg*{S_1}{X}{S_2}
  \end{array}
\end{displaymath}

Polarization:
\begin{displaymath}
  \begin{array}{l@{\quad}l@{\quad}l}
    \freety{\posty{\tau}}   & \freety*{\posty{\tau}} \\
    \freeex[\posty{\tau}]{v} & \freeex*[\posty{\tau}]{v} \\
    \fletex[\negty{\tau}][\negty{\rho}]{e}{x}{e'} & \fletex*{e}{x}{e'} \\
    \thunkty{\negty{\tau}}  & \thunkty*{\negty{\tau}} \\
    \thunkex[\negty{\tau}]{e} & \thunkex*{e} \\
    \forceex{v}   & \forceex*{v}
  \end{array}
\end{displaymath}

Parallelism:
\begin{displaymath}
  \begin{array}{l@{\quad}l@{\quad}l}
    \tensorty{\tau_1}{\tau_2} & \tensorty*{\tau_1}{\tau_2} \\
    \tensorex{v_1}{v_2} & \tensorex*{v_1}{v_2} \\
    \splitex{v}{x_1}{x_2}{e} & \splitex*{v}{x_1}{x_2}{e} \\
    \bothty{\tau_1}{\tau_2} & \bothty*{\tau_1}{\tau_2} \\
    \bothex{e_1}{e_2} & \bothex*{e_1}{e_2} \\
    \parex{e}{x}{y}{e'} & \parex*{e}{x}{y}{e'}
  \end{array}
\end{displaymath}
\end{document}

%%% Local Variables:
%%% mode: latex
%%% TeX-master: t
%%% fill-column: 90
%%% auto-fill-mode: t
%%% End:
