\documentclass[11pt]{article}
\usepackage[T1]{fontenc}
\usepackage[utf8]{inputenc}
\usepackage[english]{babel}
\usepackage{amsmath}
\usepackage{url}
\usepackage{geometry}
\usepackage{fancyvrb}

\usepackage[abt]{pl-syntax}

\title{\texttt{pl-syntax} Package%
\footnote{\copyright{} \the\year{} Robert Harper.  All Rights Reserved.}}
\author{Robert Harper}
\date{\today}

\begin{document}

\maketitle{}

\section*{Overview\footnotemark}

\footnotetext{I am grateful to Kartik Singal for help with testing and improving this package.}

Use the declaration \verb|\usepackage{pl-syntax}| to define syntax macros for programming languages inspired by the author's textbook \textit{Practical Foundations for Programming Languages}.\footnote{The former package name, \texttt{pfpl-syntax}, is denigrated, but may still be used.}  The package provides an integrated, systematic treatment of abstract and concrete syntax for a wide range of languages.

\section*{Abstract Binding Trees}

To typeset an abstract binding tree yourself, provide the package with the \verb|[abt]| option, and use the following commands:
\begin{itemize}
  \item \verb|\Opn{op}|: Format an operator as a keyword; use starred form for an operator that is to be set in math mode.  For example, \verb|\Opn{fun}| produces $\Opn{fun}$ and \verb|\Opn*{\lambda}| produces $\Opn*{\lambda}$.
  \item \verb|\Abt{operator}<parameters>[minor](major)|: produce an abt, with all arguments typeset.  In typical usage the parameters are symbols that index certain forms of abt's; the minor arguments are types that would appear in full abstract syntax; and the major arguments are abt's or abstractors constituting the components of a compound abt.  The starred form provides a ``default'' concrete syntax that omits the minor arguments.  If the parenthesized arguments are empty, then no parentheses are typeset.  For example, the command
  \begin{quote}
\begin{verbatim}
\Abt{\Opn{get}}<a>[\tau]
\end{verbatim}
  \end{quote}
  produces $\Abt{\Opn{get}}<a>[\tau]$, and the command
  \begin{quote}
\begin{verbatim}
\Abt{\Opn{set}}[\tau]<a>(e)
\end{verbatim}
  \end{quote}
  produces $\Abt{\Opn{set}}<a>[\tau](e)$.  The starred forms produce $\Abt*{\Opn{get}}<a>[\tau]$, and $\Abt*{\Opn{set}}<a>[\tau](e)$, respectively.

  \item \verb|\Abs<symbols>(variables){abt}|: produce an abstractor over symbols and/or variables within an ABT.  For example, the command
  \begin{quote}
\begin{verbatim}
\Abt{\Opn*{\lambda}}[\tau](\Abs(x){e})
\end{verbatim}
  \end{quote}
  produces $\Abt{\Opn*{\lambda}}[\tau](\Abs(x){e})$ and the starred form produces $\Abt*{\Opn*{\lambda}}[\tau](\Abs(x){e})$.  Similarly, the command
  \begin{quote}
\begin{verbatim}
\Abt{\Opn{dcl}}[\tau;\rho](e;\Abs<a>{m})
\end{verbatim}
  \end{quote}
  produces $\Abt{\Opn{dcl}}[\tau;\rho](e;\Abs<a>{m})$ and the starred form produces $\Abt*{\Opn{dcl}}[\tau\,\rho](e;\Abs<a>{m})$.

  \item \verb|\kw{abc}| sets its argument as a keyword, \verb|\sep| is the argument separator, and the operations \verb|\empMap|, \verb|\singMap{x}{y}|, \verb|\extMap{f}{x}{y}|, and \verb|\finMap{I}{i}{x}| are provided for finite maps.
\end{itemize}

The package option \verb|[sf]| indicates that keywords are to be set in sans serif font, and the package option \verb|[sc]| indicates that they are to be set in small capitals.  The package option \verb|[pfpl]| uses the notational conventions from \textsf{PFPL} in which symbol parameters to operators are typeset in square brackets, and their minor arguments are typeset in curly braces.

\smallskip

The command \verb|\Sub{e}{x}{e'}| expands to $\Sub{e}{x}{e'}$, and \verb|\Sub*{e}{x}{e'}| expands to $\Sub*{e}{x}{e'}$, providing pre- and post-fix notation for substitution.  Substitutions may be written as \verb|\empSub| for the empty substitution, $\empSub$, and \verb|\extSub{\gamma}{x}{e}| for the extension $\extSub{\gamma}{x}{e}$ of a substitution $\gamma$ sending $x$ to $e$, and \verb|\extSub*{\gamma}{x}{e}| for $\extSub*{\gamma}{x}{e}$ written in the opposite order.

\section*{Language-Specific Definitions}

The commands for typesetting the many language constructs in \textsf{PFPL}, plus some others, are organized into a collection of tables grouped roughly according to type structure.  The tables display the abstract and concrete syntax for a language construct, and give the \LaTeX{} command used to create them.  The minor arguments are generally required for the abstract syntax, but may be omitted for the concrete syntax.  For the variadic forms in Figures~\ref{fig:prod-sum}, the notation $\tau_I$, for $I$ an index set, stands for the finite map $(\finMap{I}{i}{\tau})$; the notation $e_I$ is defined analogously.

\clearpage
\newgeometry{hmargin=14mm}

\begin{figure}
  \SaveVerb{xcaseExVrb}|\xcaseEx[\tau_1][\tau_2][\rho]{e}{x_1}{e_1}{x_2}{e_2}|
  \begin{small}
    \begin{displaymath}
      \begin{array}{l@{\qquad}l@{\qquad}l@{\qquad}l}
        \textit{Abstract} & \textit{Concrete} & \textit{Invocation} \\
        \unitTy                           & \unitTy*                 & \verb|\unitTy|                                \\
        \unitEx                           & \unitEx*                 & \verb|\unitEx|                                \\
        \checkEx[\rho]{e_1}{e_2}          & \checkEx*{e_1}{e_2}       & \verb|\checkEx[\rho]{e_1}{e_2}|                \\[1ex]
        
        \prodTy{\tau_1}{\tau_2}           & \prodTy*{\tau_1}{\tau_2} & \verb|\prodTy{\tau_1}{\tau_2}|                \\
        \projEx<i>{e}                     & \projEx*<i>{e}           & \verb|\projEx<i>[\tau_1][\tau_2]{e}| \quad (i=1,2)            \\
        \pairEx[\tau_1][\tau_2]{e_1}{e_2} & \pairEx*{e_1}{e_2}       & \verb|\pairEx[\tau_1][\tau_2]{e_1}{e_2}|      \\[1ex]

        \vprodTy<I><i>{\tau}              & \vprodTy*<I><i>{\tau}          & \verb|\vprodTy<I><i>{\tau}|                   \\
        \vtupleEx<I><i>[\tau]{e}             & \vtupleEx*<I><i>{e}         & \verb|\vtupleEx<I><i>[\tau]{e}|                  \\
        \vprojEx<I><i>[\tau]{e}           & \vprojEx*<I><i>{e}          & \verb|\vprojEx<I><i>[\tau]{e}| \quad (i\in I) \\[1ex]

        \voidTy                           & \voidTy*                & \verb|\voidTy|                                    \\
        \absurdEx[\rho]{e}                & \absurdEx*{e}           & \verb|\absurdEx[\rho]{e}|                         \\[1ex]

        \sumTy{\tau_1}{\tau_2}            & \sumTy*{\tau_1}{\tau_2} & \verb|\sumTy{\tau_1}{\tau_2}|                     \\
        \sumTy<i>{\tau_1}{\tau_2}         & \sumTy*<i>{\tau_1}{\tau_2} & \verb|\sumTy<i>{\tau_1}{\tau_2}| \quad (i=1,2) \\
        \injEx<i>[\tau_1][\tau_2]{e}      & \injEx*<i>{e}           & \verb|\injEx<i>[\tau_1][\tau_2]{e}| \quad (i=1,2) \\
        \caseEx[\tau][\rho]{e}{x}{e}     & \caseEx*{e}{x}{e}      & \verb|\caseEx[\tau][\rho]{e}{x}{e}|              \\
        \xcaseEx[\tau_1][\tau_2][\rho]{e}{x_1}{e_1}{x_2}{e_2} & \xcaseEx*{e}{x_1}{e_1}{x_2}{e_2} & \\
        \multicolumn{3}{r}{\UseVerb{xcaseExVrb}} \\[1ex]

        \boolTy                           & \boolTy*                & \verb|\boolTy|                                    \\
        \trueEx                           & \trueEx*                & \verb|\trueEx|                                    \\
        \falseEx                          & \falseEx*               & \verb|\falseEx|                                   \\
        \ifEx[\rho]{e}{e_1}{e_2}          & \ifEx*{e}{e_1}{e_2}     & \verb|\ifEx[\rho]{e}{e_1}{e_2}|                   \\[1ex]

        \optTy{\tau}                      & \optTy*{\tau}           & \verb|\optTy{\tau}| \\
        \nothingEx                        & \nothingEx*             & \verb|\nothingEx| \\
        \justEx{e}                        & \justEx*{e}             & \verb|\justEx{e}| \\[2ex]

        \vsumTy<I><i>{\tau}               & \vsumTy*<I><i>{\tau}    & \verb|\vsumTy<I><i>{\tau}|                        \\
        \vinjEx<I><i>[\tau]{e}            & \vinjEx*<I><i>{e}       & \verb|\vinjEx<I><i>[\tau]{e}| \quad (i\in I)      \\
        \vcaseEx<I><i>[\tau][\rho]{e}{x}{e'} & \vcaseEx*<I><i>{e}{x}{e'}  & \verb|\vcaseEx<I><i>[\tau][\rho]{e}{x}{e'}|
      \end{array}
    \end{displaymath}
  \end{small} 

  \caption{Product and Sum Types}
  \label{fig:prod-sum}
\end{figure}

\begin{figure}
  \begin{small}
    \begin{displaymath}
      \begin{array}{l@{\qquad}l@{\qquad}l@{\qquad}l}
        \textit{Abstract} & \textit{Concrete} & \textit{Invocation} \\
        \arrTy{\tau_1}{\tau_2} & \arrTy*{\tau_1}{\tau_2} & \verb|\arrTy{\tau_1}{\tau_2}| \\
        \lamEx{x}{e}           & \lamEx*{x}{e}           & \verb|\lamEx[\tau_1][\tau_2]{x}{e}|           \\
        \appEx{e_1}{e_2}       & \appEx*{e_1}{e_2}       & \verb|\appEx[\tau_1][\tau_2]{e_1}{e_2}| \\[1ex]
        \allTy{t}{\tau}                     & \allTy*{t}{\tau}                     & \verb|\allTy{t}{\tau}|                     \\
        \tlamEx{t}[\tau]{e}                 & \tlamEx*{t}[\tau]{e}                 & \verb|\tlamEx{t}[\tau]{e}|                 \\
        \tapEx[t][\tau]{e}{\rho}          & \tapEx*{e}{\rho}                   & \verb|\tapEx[t][\tau]{e}{\rho}|          \\[1ex]
        \someTy{t}{\tau}                    & \someTy*{t}{\tau}                    & \verb|\someTy{t}{\tau}|                    \\
        \packEx[t][\tau]{\rho}{e}           & \packEx*{\rho}{e}                    & \verb|\packEx[t][\tau]{\rho}{e}|           \\
        \openEx[t][\tau][\rho]{e}{t}{x}{e'} & \openEx*[t][\tau][\rho]{e}{t}{x}{e'} & \verb|\openEx[t][\tau][\rho]{e}{t}{x}{e'}|
      \end{array}
    \end{displaymath}
  \end{small}

  \caption{Function, Universal, and Existential Types}
  \label{fig:fun-poly}
\end{figure}

\begin{figure}

  \begin{small}
    \begin{displaymath}
      \begin{array}{l@{\qquad}l@{\qquad}l@{\qquad}l}
        \textit{Abstract} & \textit{Concrete} & \textit{Invocation} \\
        \indTy{t}{\tau}                      & \indTy*{t}{\tau}                    & \verb|\indTy{t}{\tau}|                         \\
        \inEx[t][\tau]{e}                    & \inEx*{e}                           & \verb|\inEx[t][\tau]{e}|                       \\
        \recEx[t][\tau][\rho]{e}{x}{e'}      & \recEx*{e}{x}{e'}                   & \verb|\recEx[t][\tau][\rho]{e}{x}{e'}|         \\[1ex]

        \natTy                               & \natTy*                             & \verb|\natTy|                                  \\
        \zeroEx                              & \zeroEx*                            & \verb|\zeroEx|                                 \\
        \succEx{e}                           & \succEx*{e}                         & \verb|\succEx{e}|                              \\
        \natrecEx[\rho]{e}{e_0}{x}{y}{e_1}   & \natrecEx*[\rho]{e}{e_0}{x}{y}{e_1} & \verb|\natrecEx[\rho]{e}{e_0}{x}{y}{e_1}|      \\
        \natitEx[\rho]{e}{e_0}{x}{e_1}       & \natitEx*{e}{e_0}{x}{e_1}           & \verb|\natitEx[\rho]{e}{e_0}{x}{e_1}|          \\
        \ifzEx[\rho]{e}{e_0}{x}{e_1}         & \ifzEx*{e}{e_0}{x}{e_1}             & \verb|\ifzEx[\rho]{e}{e_0}{x}{e_1}|            \\[1ex]

        \listTy{\tau}                        & \listTy*{\tau}                      & \verb|\listTy{\tau}|                           \\
        \nilEx[\tau]                         & \nilEx*                             & \verb|\nilEx[\tau]|                            \\
        \consEx[\tau]{e_1}{e_2}              & \consEx*{e_1}{e_2}                  & \verb|\consEx[\tau]{e_1}{e_2}|                 \\
        \listrecEx[\rho]{e}{e_1}{x}{y}{e_2}  & \listrecEx*{e}{e_1}{x}{y}{e_2}      & \verb|\listrecEx[\rho]{e}{e_1}{x}{y}{e_2}|     \\
        \listcaseEx[\rho]{e}{e_1}{x}{y}{e_2} & \listcaseEx*{e}{e_1}{x}{y}{e_2}     & \verb|\listcaseEx[\rho]{e}{e_1}{x}{y}{e_2}|    \\[1ex]

        \coiTy{t}{\tau}                      & \coiTy*{t}{\tau}                    & \verb|\coiTy{t}{\tau}|                         \\
        \outEx[t][\tau]{e}                   & \outEx*{e}                          & \verb|\outEx[t][\tau]{e}|                      \\
        \genEx[t][\tau][\sigma]{e}{x}{e'}    & \genEx*{e}{x}{e'}                   & \verb|\genEx[t][\tau][\sigma]{e}{x}{e'}|       \\[1ex]

        \conatTy                             & \conatTy*                           & \verb|\conatTy|                                \\
        \streamTy{\tau}                      & \streamTy*{\tau}                    & \verb|\streamTy{\tau}|
      \end{array}
    \end{displaymath}
  \end{small}

  \caption{Inductive and Coinductive Types}
  \label{fig:icoi}
\end{figure}

\begin{figure}
  \SaveVerb{splitExVrb}|\splitEx[\tau_1][\tau_2][\rho]{v}{x_1}{x_2}{e}|
  \begin{small}
    \begin{displaymath}
      \begin{array}{l@{\qquad}l@{\qquad}l@{\qquad}l}
        \textit{Abstract} & \textit{Concrete} & \textit{Invocation} \\
        \tensorTy{\tau_1}{\tau_2}                      & \tensorTy*{\tau_1}{\tau_2} & \verb|\tensorTy{\tau_1}{\tau_2}|                      \\
        \tensorEx[\tau_1][\tau_2]{v_1}{v_2}            & \tensorEx*{v_1}{v_2}       & \verb|\tensorEx[\tau_1][\tau_2]{v_1}{v_2}|                            \\
        \splitEx[\tau_1][\tau_2][\rho]{v}{x_1}{x_2}{e} & \splitEx*{v}{x_1}{x_2}{e}  & \\
        \multicolumn{3}{r}{\UseVerb{splitExVrb}} \\
        \bothTy{\tau_1}{\tau_2}                        & \bothTy*{\tau_1}{\tau_2}   & \verb|\bothTy{\tau_1}{\tau_2}|                        \\
        \bothEx{e_1}{e_2}                              & \bothEx*{e_1}{e_2}         & \verb|\bothEx{e_1}{e_2}|                              \\
        \parEx{e}{x}{y}{e'}                            & \parEx*{e}{x}{y}{e'}       & \verb|\parEx{e}{x}{y}{e'}| 
      \end{array}
    \end{displaymath}
  \end{small}

  \caption{Parallel Types}
  \label{fig:par}
\end{figure}

\begin{figure}

  \begin{small}
    \begin{displaymath}
      \begin{array}{l@{\qquad}l@{\qquad}l@{\qquad}l}
        \textit{Abstract} & \textit{Concrete} & \textit{Invocation} \\
        \parrTy{\tau_1}{\tau_2}         & \parrTy*{\tau_1}{\tau_2} & \verb|\parrTy{\tau_1}{\tau_2}|          \\
        \funEx[\tau_1][\tau_2]{f}{x}{e} & \funEx*{f}{x}{e}         & \verb|\funEx[\tau_1][\tau_2]{f}{x}{e}|  \\
        \papEx{e_1}{e_2}                & \papEx*{e_1}{e_2}        & \verb|\papEx[\tau_1][\tau_2]{e_1}{e_2}| \\[1ex]

        \fixEx[\tau]{x}{e}              & \fixEx*[\tau]{x}{e}      & \verb|\fixEx[\tau]{x}{e}|               \\[1ex]

        \recTy{t}{\tau}                 & \recTy*{t}{\tau}         & \verb|\recTy{t}{\tau}|                  \\
        \foldEx[t][\tau]{e}             & \foldEx*{e}              & \verb|\foldEx[t][\tau]{e}|              \\
        \unfoldEx[t][\tau]{e}           & \unfoldEx*{e}            & \verb|\unfoldEx[t][\tau]{e}|            \\[1ex]

        \selfTy{\tau}                   & \selfTy*{\tau}           & \verb|\selfTy{\tau}|                    \\
        \unrollEx[\tau]{e}              & \unrollEx*{e}            & \verb|\unrollEx[\tau]{e}|               \\
        \selfEx[\tau]{x}{e}             & \selfEx*{x}{e}           & \verb|\selfEx[\tau]{x}{e}|              \\[1ex]

        \ulamEx{x}{M}                   & \ulamEx*{x}{M}           & \verb|\ulamEx{x}{M}|                    \\
        \uapEx{M_1}{M_2}                & \uapEx*{M_1}{M_2}        & \verb|\uapEx{M_1}{M_2}|                 \\
        \uIEx                           & \uIEx*                   & \verb|\uIEx|                            \\
        \uKEx                           & \uKEx*                   & \verb|\uKEx|                            \\
        \uSEx                           & \uSEx*                   & \verb|\uSEx|                            \\
        \uBEx                           & \uBEx*                   & \verb|\uBEx|
      \end{array}
    \end{displaymath}
  \end{small}

  \caption{Partial and Recursive Types}
  \label{fig:recursive}
\end{figure}

\begin{figure}

  \begin{small}
    \begin{displaymath}
      \begin{array}{l@{\qquad}l@{\qquad}l@{\qquad}l}
        \textit{Abstract} & \textit{Concrete} & \textit{Invocation} \\
        \ansTy                         & \ansTy*             & \verb|\ansTy|                         \\
        \yesEx                         & \yesEx*             & \verb|\yesEx|                         \\
        \noEx                          & \noEx*              & \verb|\noEx|                          \\[1ex]

        \contTy{\tau}                  & \contTy*{\tau}      & \verb|\contTy{\tau}|                  \\
        \contEx[\tau]{k}                     & \contEx*{k}         & \verb|\contEx[\tau]{k} |                    \\
        \letccEx[\tau]{x}{e}           & \letccEx*{x}{e}     & \verb|\letccEx[\tau]{x}{e}|           \\
        \throwEx[\tau][\rho]{e_1}{e_2} & \throwEx*{e_1}{e_2} & \verb|\throwEx[\tau][\rho]{e_1}{e_2}| \\[1ex]

        \empStk[\tau]                  & \empStk*[\tau]      & \verb|\empStk[\tau]|                  \\
        \extStk[\tau_1][\tau_2]{k}{f}  & \extStk*{k}{f}      & \verb|\extStk[\tau_1][\tau_2]{k}{f}|
      \end{array}
    \end{displaymath}
  \end{small}

  \caption{Continuation Types}
  \label{fig:cont}
\end{figure}

\begin{figure}

  \begin{small}
    \begin{displaymath}
      \begin{array}{l@{\qquad}l@{\qquad}l@{\qquad}l}
        \textit{Abstract} & \textit{Concrete} & \textit{Invocation} \\
        \suspTy{\tau}           & \suspTy*{\tau}           & \verb|\suspTy{\tau}|           \\
        \suspEx[\tau]{e}        & \suspEx*[\tau]{e}        & \verb|\suspEx[\tau]{e}|        \\
        \forceEx[\tau]{e}   & \forceEx*[\tau]{e}   & \verb|\forceEx[\tau]{e}|   \\[1ex]

        \compTy{\tau}           & \compTy*{\tau}           & \verb|\compTy{\tau}|           \\
        \compEx{m}              & \compEx*{m}              & \verb|\compEx{m}|              \\
        \retEx[\tau]{e}         & \retEx*[\tau]{e}         & \verb|\retEx[\tau]{e}|         \\
        \bndEx[\tau]{e_1}{x}{e_2} & \bndEx*[\tau]{e_1}{x}{e_2} & \verb|\bndEx[\tau]{e_1}{x}{e_2}|
    \end{array}
    \end{displaymath}
  \end{small}

  \caption{Suspension and Computation Types}
  \label{fig:comp-susp}
\end{figure}

\begin{figure}
  
  \begin{small}
    \begin{displaymath}
      \begin{array}{l@{\qquad}l@{\qquad}l@{\qquad}l}
        \symTy{\tau}               & \symTy*{\tau}              & \verb|\symTy{\tau}|                \\
        \quoteEx<a>                & \quoteEx*<a>               & \verb|\quoteEx<a>|                 \\
        \ifisCmd<a>[t][\tau]{e}{m_1}{m_2} & \ifisCmd*<a>{e}{m_1}{m_2}  & \verb|\ifisCmd<a>[t][\tau]{e}{m_1}{m_2}| \\
        \newsymCmd[\tau]  & \newsymCmd*[\tau] & \verb|\newsymCmd[\tau]| \\
        \ifeqCmd[t][\tau]{e_1}{e_2}{m_1}{m_2} & \ifeqCmd*{e_1}{e_2}{m_1}{m_2} & \verb|\ifeqCmd[t][\tau]{e_1}{e_2}{m_1}{m_2}| \\
      \end{array}
    \end{displaymath}
  \end{small}

  \caption{Symbols}
  \label{fig:sym}
\end{figure}

\begin{figure}

  \begin{small}
    \begin{displaymath}
      \begin{array}{l@{\qquad}l@{\qquad}l@{\qquad}l}
        \newclsCmd[\tau]              & \newclsCmd*[\tau]     & \verb|\newclsCmd[\tau]| \\
        \clsfdTy                      & \clsfdTy*             & \verb|\clsfdTy| \\
        \clsinEx<c>{e}                & \clsinEx*<c>{e}       & \verb|\clsinEx<c>{e}| \\
        \clsoutEx<c>{e}               & \clsoutEx*<c>{e}      & \verb|\clsoutEx<c>{e}| \\
        \clsrefTy{\tau}               & \clsrefTy*{\tau}      & \verb|\clsrefTy{\tau}| \\
        \clsrefEx{e}                  & \clsrefEx*{e}         & \verb|\clsrefEx{\tau}| \\
        \clsinrefEx{e_1}{e_2}         & \clsinrefEx*{e_1}{e_2} & \verb|\clsinrefEx{e_1}{e_2}| \\
        \clsoutrefEx{e_1}{e_2}        & \clsoutrefEx*{e_1}{e_2} & \verb|\clsoutrefEx{e_1}{e_2}|
      \end{array}
    \end{displaymath}
  \end{small}

  \caption{Dynamic Classification}
  \label{fig:clsfd}
\end{figure}

\begin{figure}

  \begin{small}
    \begin{displaymath}
      \begin{array}{l@{\qquad}l@{\qquad}l@{\qquad}l}
        \textit{Abstract} & \textit{Concrete} & \textit{Invocation} \\
        \cmdTy{\tau}               & \cmdTy*{\tau}               & \verb|\cmdTy{\tau}|               \\
        \cmdEx[\tau]{m}            & \cmdEx*{m}                  & \verb|\cmdEx[\tau]{m}|            \\[1ex]
        \retCmd[\tau]{e}           & \retCmd*{e}                 & \verb|\retCmd[\tau]{e}|           \\
        \raiseCmd[\rho]{e}         & \raiseCmd*{e}               & \verb|\raiseCmd[\rho]{e}| \\
        \bndCmd[\tau]{e}[\rho]{x}{m}           & \bndCmd*{e}{x}{m}           & \verb|\bndCmd[\tau]{e}[\rho]{x}{m}|           \\
        \bndowCmd[\tau]{e}{x}{m_1}{x}{m_2}     & \bndowCmd*{e}{x}{m_1}{x}{m_2}  & \verb|\bndowCmd[\tau]{e}{x}{m_1}{x}{m_2}| \\
        \dclCmd[\tau]{e}[\rho]<a>{m}           & \dclCmd*{e}<a>{m}           & \verb|\dclCmd[\tau]{e}[\rho]<a>{m}|           \\[1ex]
        \retunitCmd{}              & \retunitCmd*{}              & \verb|\retunitCmd| \\
        \doCmd[\tau]{m}            & \doCmd*{m}                  & \verb|\doCmd[\tau]{m}| \\
        \seqCmd{m_1}[x]{m_2}       & \seqCmd*{m_1}[x]{m_2}       & \verb|\seqCmd{m_1}[x]{m_2}| \\
        \seqCmd{m_1}{m_2}          & \seqCmd*{m_1}{m_2}          & \verb|\seqCmd{m_1}{m_2}| \\[1ex]
        \refTy{\tau}               & \refTy*{\tau}               & \verb|\refTy{\tau}| \\
        \refEx<a>                  & \refEx*<a>                  & \verb|\refEx<a>|                  \\[1ex]
        \getCmd<a>                 & \getCmd*<a>                 & \verb|\getCmd<a>|                 \\
        \getrefCmd[\tau]{e}        & \getrefCmd*{e}              & \verb|\getrefCmd[\tau]{e}|        \\
        \setCmd<a>{e}              & \setCmd*<a>{e}              & \verb|\setCmd<a>{e}|              \\
        \setrefCmd[\tau]{e_1}{e_2} & \setrefCmd*[\tau]{e_1}{e_2} & \verb|\setrefCmd[\tau]{e_1}{e_2}|  \\[1ex]
        \newrefCmd[\tau]{e}        & \newrefCmd*{e}              & \verb|\newrefCmd[\tau]{e}| \\[1ex]
        \preincCmd<a>              & \preincCmd*<a>              & \verb|\preincCmd<a>| \\
        \postincCmd<a>             & \postincCmd*<a>             & \verb|\postincCmd<a>| \\
        \predecCmd<a>              & \predecCmd*<a>              & \verb|\predecCmd<a>| \\
        \postdecCmd<a>             & \postdecCmd*<a>             & \verb|\postdecCmd<a>|
      \end{array}
    \end{displaymath}
  \end{small}

  \caption{Command Types}
  \label{fig:cmd}
\end{figure}

\begin{figure}

  \begin{small}
    \begin{displaymath}
      \begin{array}{l@{\qquad}l@{\qquad}l@{\qquad}l}
        \textit{Abstract} & \textit{Concrete} & \textit{Invocation} \\
        \staticLock                    & \staticLock*         & \verb|\staticLock|                              \\
        \univSg                        & \univSg*             & \verb|\univSg|                                  \\
        \valSg{\tau}                   & \valSg*{\tau}        & \verb|\valSg{\tau}|                             \\[1ex]
        \extSg{S}{M}                   & \extSg*{S}{M}        & \verb|\extSg{S}{M}|                             \\
        \inMd[S][M]{M'}                & \inMd*{M'}           & \verb|\inMd[S][M]{M'}|                          \\
        \outMd[S][M]{M'}               & \outMd*{M'}          & \verb|\outMd[S][M]{M'}|                         \\[1ex]
        \compSg{S}                     & \compSg*{S}          & \verb|\compSg{S}|                               \\[1ex]
        \piSg{S_1}[X]{S_2}             & \piSg*{S_1}[X]{S_2}  & \verb|\piSg{S_1}[X]{S_2}|                       \\
        \piSg{S_1}{S_2}                & \piSg*{S_1}{S_2}     & \verb|\piSg{S_1}{S_2}| \\
        \funMd{S_1}{X}{S_2}{M_2}       & \funMd*{S_1}{X}{S_2}{M_2} & \verb|\funMd{S_1}{X}{S_2}{M_2}|                      \\
        \instMd[S_1][X][S_2]{M_1}{M_2} & \instMd*{M_1}{M_2}   & \verb|\instMd[S_1][X][S_2]{M_1}{M_2}|           \\[1ex]
        \sigSg{S_1}[X]{S_2}            & \sigSg*{S_1}[X]{S_2} & \verb|\sigSg{S_1}[X]{S_2}|                      \\
        \sigSg{S_1}{S_2}               & \sigSg*{S_1}{S_2}    & \verb|\sigSg{S_1}{S_2}| \\
        \strMd[S_1][X][S_2]{M_1}{M_2}  & \strMd*{M_1}{M_2}    & \verb|\strMd[S_1][X][S_2]{M_1}{M_2}|            \\
        \projMd<i>[S_1][X][S_2]{M}     & \projMd*<i>{M}       & \verb|\projMd<i>[S_1][X][S_2]{M}| \quad (i=1,2)
      \end{array}
    \end{displaymath}
  \end{small}

  \caption{Module Types}
  \label{fig:mod}
\end{figure}

\begin{figure}

  \begin{small}
    \begin{displaymath}
      \begin{array}{l@{\qquad}l@{\qquad}l@{\qquad}l}
        \textit{Abstract} & \textit{Concrete} & \textit{Invocation} \\
        \unitPr            & \unitPr*            & \verb|\unitPr |           \\
        \parPr{p_1}{p_2}     & \parPr*{p_1}{p_2}     & \verb|\parPr{p_1}{p_2}|     \\[1ex]
        \nullPr            & \nullPr*            & \verb|\nullPr|            \\
        \choosePr{p_1}{p_2}  & \choosePr*{p_1}{p_2}  & \verb|\choosePr{p_1}{p_2}|  \\[1ex]
        \quePr{a}{p}         & \quePr*{a}{p}         & \verb|\quePr{a}{p}|         \\
        \sigPr{a}{p}         & \sigPr*{a}{p}         & \verb|\sigPr{a}{p}|                     \\[1ex]
        \rcvPr{a}{x}{p}      & \rcvPr*{a}{x}{p}      & \verb|\rcvPr{a}{x}{p}|         \\
        \sndPr{a}{e}{p}      & \sndPr*{a}{e}{p}      & \verb|\sndPr{a}{e}{p}|      \\[1ex]
        \asndPr{a}{e}        & \asndPr*{a}{e}        & \verb|\asndPr{a}{e}|        \\
        \syncPr{e}          & \syncPr*{e} & \verb|\syncPr{e}| \\[1ex]
        \newPr[\tau]{a}{p}         & \newPr*{a}{p}         & \verb|\newPr[\tau]{a}{p}|         \\[1ex]
        \silAc               & \silAc*               & \verb|\silAc|               \\
        \sigAc{a}            & \sigAc*{a}            & \verb|\sigAc{a}|            \\
        \queAc{a}            & \queAc*{a}            & \verb|\queAc{a}|            \\
        \sndAc{a}{e}         & \sndAc*{a}{e}         & \verb|\sndAc{a}{e}|         \\
        \rcvAc{a}{e}         & \rcvAc*{a}{e}         & \verb|\rcvAc{a}{e}|         \\
        \anyAc{e}            & \anyAc*{e}            & \verb|\anyAc{e}|         \\
        \emitAc{e}           & \emitAc*{e}           & \verb|\emitAc{e}|
      \end{array}
    \end{displaymath}
  \end{small}

  \caption{Processes}
  \label{fig:proc}
\end{figure}

\begin{figure}
  \begin{small}
    \begin{displaymath}
      \begin{array}{l@{\qquad}l@{\qquad}l@{\qquad}l}
        \emitCmd[\tau]{e}     & \emitCmd*{e}        & \verb|\emitCmd[\tau]{e}| \\
        \accCmd[\tau]         & \accCmd*{}          & \verb|\accCmd[\tau]| \\
        \syncCmd[\tau]{e}     & \syncCmd*{e}        & \verb|\syncCmd[\tau]{e}| \\
        \spawnCmd[\tau]{e}    & \spawnCmd*{e}       & \verb|\spawnCmd[\tau]{e}| \\
        \newchCmd{\tau}       & \newchCmd*{\tau}    & \verb|\newchCmd{\tau}| \\
        \sndCmd<a>{e}         & \sndCmd*<a>{e}      & \verb|\sndCmd<a>{e}| \\
        \rcvCmd<a>{x}{e}      & \rcvCmd*<a>{x}{e}   & \verb|\rcvCmd<a>{x}{e}| \\[1ex]
        \rcvEv<a>             & \rcvEv*<a>          & \verb|\rcvEv<a>| \\
        \sndEv<a>{v}          & \sndEv*<a>{v}       & \verb|\sndEv<a>| \\
        \orEv[\tau]{e_1}{e_2}       & \orEv*{e_1}{e_2}    & \verb|\orEv[\tau]{e_1}{e_2}| \\
        \wrapEv[\tau]{e_1}{x}{e_2}  & \wrapEv*{e_1}{x}{e_2} & \verb|\wrapEv[\tau]{e_1}{x}{e_2}|
      \end{array}
    \end{displaymath}
  \end{small}
  \caption{Concurrent Algol}
  \label{fig:conc}
\end{figure}

\begin{figure}

  % \SaveVerb{copletExVrb}|\copletEx[\tau_1][\tau_2][\rho]{v}{x_1}{x_2}{e}|
  \begin{small}
    \begin{displaymath}
      \begin{array}{l@{\qquad}l@{\qquad}l@{\qquad}l}
        \textit{Abstract} & \textit{Concrete} & \textit{Invocation} \\
        \topTy                                & \topTy*                   & \verb|\topTy| \\
        \topEx                                & \topEx*                   & \verb|\topEx| \\
        \chckEx[\rho]{e}{e'}                  & \chckEx*{e}{e'}          & \verb|\checkEx[\rho]{e}{e'}| \\
        \freeTy{\tau}                         & \freeTy*{\tau}    & \verb|\freeTy{\tau}|                         \\
        \freeEx[\tau]{v}                      & \freeEx*[\tau]{v} & \verb|\freeEx[\tau]{v}|                      \\
        \fletEx[\tau][\rho]{e}{x}{e'} & \fletEx*{e}{x}{e'}        & \verb|\fletEx[\tau][\rho]{e}{x}{e'}| \\
        \thunkTy{\tau}                        & \thunkTy*{\tau}   & \verb|\thunkTy{\tau}|                        \\
        \thunkEx[\tau]{e}                     & \thunkEx*{e}              & \verb|\thunkEx[\tau]{e}|                     \\
        \recthunkEx<k>[\tau]{x}{e}           & \recthunkEx*<k>[\tau]{x}{e}  & \verb|\recthunkEx<k>[\tau]{x}{e}|   \\
        \forceEx[\tau]{v}                     & \forceEx*{v}              & \verb|\forceEx[\tau]{v}| %\\
        % \copTy{\tau_1}{\rho_2}            & \copTy*{\tau_1}{\tau_2}     & \verb|\copTy{\tau_1}{\tau_2}| \\
        % \copEx[\tau_1][\tau_2]{v_1}{e_2} & \copEx*{v_1}{e_2}           & \verb|\copEx[\tau_1][\tau_2]{v_1}{e_2}| \\
        % \copletEx{v}{x_1}{x_2}{e}                       & \copletEx*{v}{x_1}{x_2}{e}  & \\
        %   \multicolumn{3}{r}{\UseVerb{copletExVrb}}
      \end{array}
    \end{displaymath}
  \end{small}

  \caption{Polarized Types}
  \label{fig:pol}
\end{figure}

\SaveVerb{orelVrb}|\orEPf[\phi_1][\phi_2][\rho]{\pi}{x}{\pi_1}{x}{\pi_2}|
\begin{figure}

  \begin{small}
    \begin{displaymath}
      \begin{array}{l@{\qquad}l@{\qquad}l@{\qquad}l}
        \textit{Abstract} & \textit{Concrete} & \textit{Invocation} \\
        \trueProp                                       & \trueProp*                       & \verb|\trueProp|                                       \\
        \trueIPf                                        & \trueIPf*                        & \verb|\trueIPf|                                        \\[1ex]

        \andProp{\phi_1}{\phi_2}                        & \andProp*{\phi_1}{\phi_2}        & \verb|\andProp{\phi_1}{\phi_2}|                        \\
        \andIPf[\phi_1][\phi_2]{\pi_1}{\pi_2}                           & \andIPf*{\pi_1}{\pi_2}           & \verb|\andIPf[\phi_1][\phi_2]{\pi_1}{\pi_2}|                           \\
        \andEPf[\phi_1][\phi_2]<i>{\pi}                 & \andEPf*<i>{\pi}                 & \verb|\andEPf[\phi_1][\phi_2]<i>{\pi}|                 \\[1ex]

        \falseProp                                      & \falseProp*                      & \verb|\falseProp|                                      \\
        \falseIPf[\phi]{\pi}                            & \falseIPf*{\pi}                  & \verb|\falseIPf[\phi]{\pi}|                            \\
        \falseEPf[\phi]{\pi}                            & \falseEPf*{\pi}                  & \verb|\falseEPf[\phi]{\pi}|                            \\[1ex]

        \orProp{\phi_1}{\phi_2}                         & \orProp*{\phi_1}{\phi_2}         & \verb|\orProp{\phi_1}{\phi_2}|                         \\
        \orIPf[\phi_1][\phi_2]<i>{\pi}                  & \orIPf*<i>{\pi}                  & \verb|\orIPf[\phi_1][\phi_2]<i>{\pi}|                 \\
        \orEPf[\phi_1][\phi_2][\rho]{\pi}{x}{\pi_1}{x}{\pi_2} &
        \orEPf*{\pi}{x}{\pi_1}{x}{\pi_2} & \\
        \multicolumn{3}{r}{\UseVerb{orelVrb}} \\[1ex]

        \impProp{\phi_1}{\phi_2}                        & \impProp*{\phi_1}{\phi_2}        & \verb|\impProp{\phi_1}{\phi_2}|                        \\
        \impIPf[\phi_1][\phi_2]{x}{\pi_2}               & \impIPf*{x}{\pi_2}               & \verb|\impIPf[\phi_1][\phi_2]{x}{\pi_2}|               \\
        \impEPf[\phi_1][\phi_2]{\pi}{\pi_1}             & \impEPf*{\pi}{\pi_2}             & \verb|\impEPf[\phi_1][\phi_2]{\pi}{\pi_1}|             \\[1ex]

        \notProp{\phi}                                  & \notProp*{\phi}                  & \verb|\notProp{\phi}|                                  \\
        \notIPf[\phi]{x}{\pi}                           & \notIPf*{x}{\pi}                 & \verb|\notIPf[\phi]{x}{\pi}|                           \\
        \notEPf[\phi]{\pi}{\pi_2}                       & \notEPf*{\pi}{\pi_2}             & \verb|\notEPf[\phi]{\pi}{\pi_2}|
      \end{array}
    \end{displaymath}
  \end{small}

  \caption{Propositions and Proofs}
  \label{fig:prop}
\end{figure}

\restoregeometry

\end{document}
